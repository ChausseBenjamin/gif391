\documentclass[a11paper, 11pt]{article}

\usepackage{document}
\usepackage{titlepage}
\usepackage[T1]{fontenc}
\usepackage[french]{babel}
\usepackage{codeblocks}
%\usepackage{xcolor}

% \addbibresource{bibliography.bib}
% \nofiles

\newcommand{\todo}[1]{\textcolor{orange}{\textbf{TODO}: #1}}
\newcommand{\note}[1]{\textcolor{purple}{\textbf{NOTE}: #1}}
\newcommand{\xxx}[1]{\textcolor{red}{\textbf{XXX}: #1}}

% \institution{Université de Sherbrooke}
% \faculty{Faculté de génie}
% \department{Département de génie électrique et de génie informatique}
\title{Rapport d'APP}
\classnb{GIF391}
\class{Conception d'un système distribué}
\author{
  \addtolength{\tabcolsep}{-0.4em}
  \begin{tabular}{rcl} % Ajouter des auteurs au besoin
  Benjamin Chausse & -- & CHAB1704 \\
  Samuel Bilodeau  & -- & BILS2704 \\
  \end{tabular}
}
\teacher{Frédéric Mailhot}
% \location{Sherbrooke}
% \date{\today}

\begin{document}
\maketitle
\newpage
\tableofcontents
\newpage

\section{Description des solutions utilisées}

\section{Discussion de la structure}

La version finale du système offre une plus grande flexibilité par rapport aux premières itérations. Son principal inconvénient réside dans sa complexité croissante, ce qui peut entraîner des problèmes à différents niveaux, parfois difficiles à détecter. Cependant, les avantages sont nombreux. La nature distribuée du système facilite son expansion horizontale, le rendant ainsi adapté à un grand nombre d'utilisateurs. L'utilisation d'une base de données garantit l'atomicité des transactions pour éviter les conflits et les enregistrements en double. La réplication des serveurs et des bases de données augmente la redondance et, par conséquent, la résilience du système. La modularité des composants permet des mises à niveau fluides et transparentes pour l'utilisateur.

\subsection{Technologies sous-jacentes}

\subsubsection{Identification des ressources}

La technologie utilisée pour identifier les ressources s'appelle "namespace". Elle nous a permis d'attribuer différentes ressources. Au début, nous l'avons principalement utilisée pour assigner des emplacements de stockage afin de partager des fichiers entre les différents conteneurs. Par la suite, lorsque des requêtes réseau via SSH ont été introduites, nous avons utilisé les namespaces pour créer un réseau virtuel sur lequel les machines pouvaient communiquer facilement. Les noms des conteneurs étaient utilisés pour remplacer les adresses IP, ce qui simplifiait grandement nos interactions. En résumé, les namespaces sont utiles pour assigner diverses ressources telles que le réseau, les systèmes de fichiers, les processus, les groupes ou les utilisateurs, et bien d'autres encore.

\subsubsection{Contrôle d'accès aux ressources}

Pour contrôler l'accès aux ressources, nous avons utilisé les cgroups. Ces derniers permettent d'attribuer l'allocation de différentes ressources (processeur, mémoire, disque) en fonction des membres du cgroup. Ces membres peuvent être un processus spécifique, un utilisateur ou un conteneur dans son intégralité. Nous pouvons gérer la quantité minimale de ressources et donner la priorité à certaines entités par rapport à d'autres. Les cgroups sont également un bon moyen de gérer les accès à des répertoires et à des réseaux spécifiques.

\subsubsection{Gestion des accès aux fichiers utilisés}


Docker utilise une technologie Linux très pratique appelée "système de fichiers en couches" (Union File System). Le concept est simple : certains répertoires sont en lecture seule, et lorsqu'une modification est nécessaire, elle est enregistrée dans un répertoire qui stocke uniquement les différences (un peu comme la technologie Git). Cette approche permet à plusieurs conteneurs d'utiliser la même base commune, mais chacun d'entre eux de manière distincte avec son propre répertoire de modifications. Dans la dernière itération, plusieurs clients utilisaient l'image client, mais chacun était unique. Cela nous permet d'utiliser efficacement l'espace de stockage et d'éviter de corrompre l'image initiale.


\todo{Quels sont les pilotes utilisables pour la persistance du système de fichiers
à union ?} \\
\todo{Quel pilote de persistence a été utilisé et pourquoi ?}

\subsubsection{Configuration réseau pour la communication des conteneurs}

Pour simplifier l'environnement dans lequel nous travaillions, nous avons utilisé un namespace de réseau pour regrouper toutes les entités sur la même couche. Bien que cela fonctionnait, cette architecture n'était pas la plus idéale dans un environnement de production. Certaines entités, comme les clients, ne devraient pas nécessairement avoir accès aux autres, telles que la base de données. En général, ceux-ci sont disposées sur des réseaux distants et les interactions requises sont déterminées à l'avance. Afin d'assurer l'encapsulation des composants et la sécurité du système, l'authentification et le chiffrement sont des étapes obligatoires avant toute action. Étant donné que l'objectif de l'activité ne portait pas spécifiquement sur la mise en réseau, tous les services étaient visibles et accessibles à tous.

\subsubsection{Duplication mise en place pour les ressources}

\todo{Quelles ressources doivent être dupliquées, et pourquoi?} \\
\todo{Dans quel cas la duplication peut se faire sur une machine réelle unique, dans
quel cas elle doit être distribuée sur plusieurs machines réelles ?}


% \newpage
% \printbibliography[heading=bibintoc]
\end{document}
