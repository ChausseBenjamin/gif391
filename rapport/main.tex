\documentclass[a11paper, 11pt]{article}

\usepackage{document}
\usepackage{titlepage}
\usepackage[T1]{fontenc}
\usepackage[french]{babel}
\usepackage{codeblocks}
\usepackage{xcolor}

% \addbibresource{bibliography.bib}
% \nofiles

\newcommand{\todo}[1]{\textcolor{orange}{\textbf{TODO}: #1}}
\newcommand{\note}[1]{\textcolor{purple}{\textbf{NOTE}: #1}}
\newcommand{\xxx}[1]{\textcolor{red}{\textbf{XXX}: #1}}

% \institution{Université de Sherbrooke}
% \faculty{Faculté de génie}
% \department{Département de génie électrique et de génie informatique}
\title{Rapport d'APP}
\classnb{GIF391}
\class{Conception d'un système distribué}
\author{
  \addtolength{\tabcolsep}{-0.4em}
  \begin{tabular}{rcl} % Ajouter des auteurs au besoin
  Benjamin Chausse & -- & CHAB1704 \\
  Samuel Bilodeau  & -- & BILS2704 \\
  \end{tabular}
}
\teacher{Frédéric Mailhot}
% \location{Sherbrooke}
% \date{\today}

\begin{document}
\maketitle
\newpage
\tableofcontents
\newpage

\section{Description des solutions utilisées}

\section{Discussion de la structure}

\todo{Quelle est la structure du système distribué de façon générique?} \\

\subsection{Technologies sous-jacentes}

\subsubsection{Identification des ressources}

\subsubsection{Contrôle d'accès aux ressources}

\subsubsection{Gestion des accès aux fichiers utilisés}

\todo{Quels sont les pilotes utilisables pour la persistance du système de fichiers
à union ?} \\
\todo{Quel pilote de persistence a été utilisé et pourquoi ?}

\subsubsection{Configuration réseau pour la communication des conteneurs}

\subsubsection{Duplication mise en place pour les ressources}

\todo{Quelles ressources doivent être dupliquées, et pourquoi?} \\
\todo{Dans quel cas la duplication peut se faire sur une machine réelle unique, dans
quel cas elle doit être distribuée sur plusieurs machines réelles ?}


% \newpage
% \printbibliography[heading=bibintoc]
\end{document}
